\documentclass[10pt, a4paper, twoside]{basestyle}

\usepackage{tikz}
\usetikzlibrary{cd}

\usepackage[Mathematics]{semtex}
\usepackage{chngcntr}
\counterwithout{equation}{section}

%%%% Shorthands.
\DeclareMathOperator{\bias}{\mathit{bias}}
\DeclareMathOperator{\ULP}{ulp}

% Rounding brackets will be heavily nested, and reading the nesting depth is critically important,
% so we make them grow for readability.
\newcommand{\round}[1]{\doubleSquareBrackets*{#1}}
\newcommand{\roundTowardZero}[1]{\doubleSquareBrackets{#1}_0}
\newcommand{\roundTowardPositive}[1]{\doubleSquareBrackets{#1}_+}
\newcommand{\roundTowardNegative}[1]{\doubleSquareBrackets{#1}_-}
\newcommand{\hex}[1]{{_{16}}\mathrm{#1}}
\newcommand{\bin}[1]{{_{2}}\mathrm{#1}}

%%%% Title and authors.

\title{An Implementation of Sin and Cos Using Gal's Accurate Tables}
\date{\printdate{2025-02-02}}
\author{Pascal~Leroy (phl)}
\begin{document}
\maketitle
\begin{sloppypar}
\noindent
This document describes the implementation of functions \texttt{Sin} and \texttt{Cos} in Principia.  The goals of that implementation are to be portable (including to machines that do not have a fused multiply-add instruction), achieve good performance, and ensure correct rounding.
\end{sloppypar}

\section*{Overview}
The implementation follows the ideas described by \cite{GalBachelis1991} and uses accurate tables produced by the method presented in \cite{StehléZimmermann2005}.  It guarantees correct rounding with a high probability.  In circumstances where it cannot guarantee correct rounding, it falls back to the (slower but correct) implementation provided by the CORE-MATH project \cite{SibidanovZimmermannGlondu2022} \cite{ZimmermannSibidanovGlondu2024}.  More precisely, the algorithm proceeds through the following steps:
\begin{itemize}[nosep]
\item perform argument reduction using Cody and Waite's algorithm in double precision (see \cite[379]{MullerBrisebarreDeDinechinJeannerodLefevreMelquiondRevolStehleTorres2010});
\item if argument reduction loses too many bits (i.e., the argument is close to a multiple of $\frac{\gp}{2}$), fall back to \texttt{cr\_sin} or \texttt{cr\_cos};
\item otherwise, uses accurate tables and a polynomial approximation to compute \texttt{Sin} or \texttt{Cos} with extra accuracy;
\item if the result has a ``dangerous rounding configuration'' (as defined by \cite{GalBachelis1991}), fall back to \texttt{cr\_sin} or \texttt{cr\_cos};
\item otherwise return the rounded result of the preceding computation.
\end{itemize}
In this document we assume a base-2 floating-point number system similar to IEEE \texttt{binary64}.  In a system with $M$ mantissa bits\footnote{In \texttt{binary64}, $M = 53$.}, the floating-point number $x$ is represented by a pair of integers $(m_x, e_x)$ such that:
\[
x = ±m_x \times 2^{e_x} \qquad\text{with}\qquad 2^{M-1} \leq m_x \leq 2^M - 1
\]
In this system, the distance between $1$ and the next larger floating-point number is:
\[
\ge_M \DefineAs 2^{1-M}
\]
and the distance between $1$ and the next smaller floating-point number is $\frac{\ge_M}{2}$.
The \textit{unit of the last place} of $x$ is defined as:
\[
\ULP\of{x} \DefineAs 2^{e_x}
\]
In particular, $\ULP\of{1} = \ge_M$.

Note that we ignore the exponent bias, overflow and underflow as they play no role in this document.
\section*{Argument Reduction}
The argument $x$ to \texttt{Sin} and \texttt{Cos} is reduced to a pair $(\hat x, \hat{\gd x})$ such that:
\[
\begin{dcases}
\hat x + \hat{\gd x}≅ n \frac{\gp}{2} + x \\
\hat x \in \intclos{-\frac{\gp}{4}}{\frac{\gp}{4}} \\
\abs{\hat{\gd x}} < \ULP\of{\hat x} 
\end{dcases}
\]
The reduction proceeds as follows:
\begin{itemize}[nosep]
\item if $\abs x < \frac{\gp}{4}$ then $\hat x = x$ and $\hat{\gd x} = 0$;
\end{itemize}
\section*{Accurate Tables and Their Generation}
\section*{Computation of the Functions}

\end{document}
