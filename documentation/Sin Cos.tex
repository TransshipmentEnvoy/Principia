\documentclass[10pt, a4paper, twoside]{basestyle}

\usepackage{tikz}
\usetikzlibrary{cd}

\usepackage[Mathematics]{semtex}
\usepackage{chngcntr}
\counterwithout{equation}{section}

%%%% Shorthands.
\DeclareMathOperator{\bias}{\mathit{bias}}
\DeclareMathOperator{\ULP}{\mathfrak u}
\DeclareMathOperator{\mant}{\mathfrak m}
\DeclareMathOperator{\expn}{\mathfrak e}

% Rounding brackets will be heavily nested, and reading the nesting depth is critically important,
% so we make them grow for readability.
\newcommand{\round}[1]{\doubleSquareBrackets*{#1}}
\newcommand{\roundTowardZero}[1]{\doubleSquareBrackets{#1}_0}
\newcommand{\roundTowardPositive}[1]{\doubleSquareBrackets{#1}_+}
\newcommand{\roundTowardNegative}[1]{\doubleSquareBrackets{#1}_-}
\newcommand{\hex}[1]{{_{16}}\mathrm{#1}}
\newcommand{\bin}[1]{{_{2}}\mathrm{#1}}

%%%% Title and authors.

\title{An Implementation of Sin and Cos Using Gal's Accurate Tables}
\date{\printdate{2025-02-02}}
\author{Pascal~Leroy (phl)}
\begin{document}
\maketitle
\begin{sloppypar}
\noindent
This document describes the implementation of functions \texttt{Sin} and \texttt{Cos} in Principia.  The goals of that implementation are to be portable (including to machines that do not have a fused multiply-add instruction), achieve good performance, and ensure correct rounding.
\end{sloppypar}

\section*{Overview}
The implementation follows the ideas described by \cite{GalBachelis1991} and uses accurate tables produced by the method presented in \cite{StehléZimmermann2005}.  It guarantees correct rounding with a high probability.  In circumstances where it cannot guarantee correct rounding, it falls back to the (slower but correct) implementation provided by the CORE-MATH project \cite{SibidanovZimmermannGlondu2022} \cite{ZimmermannSibidanovGlondu2024}.  More precisely, the algorithm proceeds through the following steps:
\begin{itemize}[nosep]
\item perform argument reduction using Cody and Waite's algorithm in double precision (see \cite[379]{MullerBrisebarreDeDinechinJeannerodLefevreMelquiondRevolStehleTorres2010});
\item if argument reduction loses too many bits (i.e., the argument is close to a multiple of $\frac{\Pi}{2}$), fall back to \texttt{cr\_sin} or \texttt{cr\_cos};
\item otherwise, uses accurate tables and a polynomial approximation to compute \texttt{Sin} or \texttt{Cos} with extra accuracy;
\item if the result has a ``dangerous rounding configuration'' (as defined by \cite{GalBachelis1991}), fall back to \texttt{cr\_sin} or \texttt{cr\_cos};
\item otherwise return the rounded result of the preceding computation.
\end{itemize}
In this document we assume a base-2 floating-point number system with $M$ significand bits\footnote{In \texttt{binary64}, $M = 53$.} similar to the IEEE formats.  We define a real  function $\mant$ and an integer function $\expn$ denoting the \emph{significand} and \emph{exponent} of a real number, respectively:
\[
x = ±\mant\of{x} \times 2^{\expn\of{x}} \qquad\text{with}\qquad 2^{M-1} \leq \mant\of{x} \leq 2^M - 1
\]
Note that this representation is unique.  Furthermore, if $x$ is a floating-point number, $\mant\of{x}$ is an integer.

The distance between $1$ and the next larger floating-point number is:
\[
\ge_M \DefineAs 2^{1-M}
\]
and the distance between $1$ and the next smaller floating-point number is $\frac{\ge_M}{2}$.
The \emph{unit of the last place} of $x$ is defined as:
\[
\ULP\of{x} \DefineAs 2^{\expn\of{x}}
\]
In particular, $\ULP\of{1} = \ge_M$.

We ignore the exponent bias, overflow and underflow as they play no role in this discussion.
\section*{Argument Reduction}
Given an argument $x$, the purpose of argument reduction is to compute a pair of floating-point numbers $\pa{\hat x, \gd \hat x}$ such that:
\[
\begin{dcases}
\hat x + \gd \hat x ≅ x \pmod{\frac{\Pi}{2}} \\
\hat x \;\text{is approximately in}\; \intclos{-\frac{\Pi}{4}}{\frac{\Pi}{4}} \\
\abs{\gd \hat x} < \ULP\of{\hat x} 
\end{dcases}
\]
\subsection*{Approximation of $\Pi$}
We approximate $\frac{\Pi}{2}$ as the sum of two floating-point numbers:
\[
\frac{\Pi}{2} ≅ C + \gd C
\]
where $C$ is obtained by truncating $\frac{\Pi}{2}$ to $\gk_1$ significand bits:
\[
C \DefineAs \floor{2^{-\gk_1} \mant \of{\frac{\Pi}{2}}} 2^{\gk_1} \ULP\of{\frac{\Pi}{2}}
\]
and $\gd C$ is defined as $\round{\frac{\Pi}{2} - C}$.  Obviously we have:
\[
0 < \frac{\Pi}{2} - C < 2^{\gk_1} \ULP\of{\frac{\Pi}{2}}
\]
but if $\gk_1$ is chosen to cut the significand of $\frac{\Pi}{2}$ at a place where it has zeroes, we can actually have a stricter bound:
\[
\frac{\Pi}{2} - C < 2^{\gk_2} \ULP\of{\frac{\Pi}{2}} \qquad\text{with}\qquad \gk_2 \leq \gk_1
\]
and therefore:
\[
\ULP\of{\frac{\Pi}{2} - C} < \frac{2^{\gk_2} \ULP\of{\frac{\Pi}{2}}}{\mant\of{\frac{\Pi}{2} - C}} \leq 2^{\gk_2 - M + 1} \ULP\of{\frac{\Pi}{2}} 
\]
Since the function $\ULP$ is always a power of $2$ this implies:
\[
\ULP\of{\frac{\Pi}{2} - C} = 2^{\gk_2 -M} \ULP\of{\frac{\Pi}{2}}
\]
and:
\[
\abs{\frac{\Pi}{2} - C - \gd C} \leq \frac{1}{2} \ULP\of{\frac{\Pi}{2} - C} = 2^{\gk_2 - M - 1} \ULP\of{\frac{\Pi}{2}}
\]
In other words, we have a representation with a significand that has effectively $2 M - \gk_2$ bits and is such that multiplying $C$ by an integer less than or equal to $2^{\gk_1}$ is exact.  The representation of $\frac{\Pi}{2}$ has three zeroes after the 18th bit of its significand, so by taking $\gk_1 = 18$ we have $\gk_2 = 14$.

\subsection*{Argument Reduction for Small Angles}
If $\abs x < \round{\frac{\Pi}{4}}$ then $\hat x = x$ and $\gd \hat x = 0$.
\subsection*{Argument Reduction for Medium Angles}
If $\abs x < \round{2^{\gk_1} \frac{\Pi}{2}}$ then we compute:
\[
n = \round{x \round{\frac{2}{\Pi}}
\]
\subsection*{Argument Reduction for Large Angles}
\section*{Accurate Tables and Their Generation}
\section*{Computation of the Functions}

\end{document}
