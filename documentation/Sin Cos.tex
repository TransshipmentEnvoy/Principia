\documentclass[10pt, a4paper, twoside]{basestyle}

\usepackage{tikz}
\usetikzlibrary{cd}

\usepackage[Mathematics]{semtex}
\usepackage{chngcntr}
\counterwithout{equation}{section}

%%%% Shorthands.
\DeclareMathOperator{\bias}{\mathit{bias}}
\DeclareMathOperator{\ULP}{\mathfrak u}
\DeclareMathOperator{\mant}{\mathfrak m}
\DeclareMathOperator{\expn}{\mathfrak e}

% Rounding brackets will be heavily nested, and reading the nesting depth is critically important,
% so we make them grow for readability.
\newcommand{\round}[1]{\doubleSquareBrackets*{#1}}
\newcommand{\roundTowardZero}[1]{\doubleSquareBrackets{#1}_0}
\newcommand{\roundTowardPositive}[1]{\doubleSquareBrackets{#1}_+}
\newcommand{\roundTowardNegative}[1]{\doubleSquareBrackets{#1}_-}
\newcommand{\hex}[1]{{_{16}}\mathrm{#1}}
\newcommand{\bin}[1]{{_{2}}\mathrm{#1}}

%%%% Title and authors.

\title{An Implementation of Sin and Cos Using Gal's Accurate Tables}
\date{\printdate{2025-02-02}}
\author{Pascal~Leroy (phl)}
\begin{document}
\maketitle
\begin{sloppypar}
\noindent
This document describes the implementation of functions \texttt{Sin} and \texttt{Cos} in Principia.  The goals of that implementation are to be portable (including to machines that do not have a fused multiply-add instruction), achieve good performance, and ensure correct rounding.
\end{sloppypar}

\section*{Overview}
The implementation follows the ideas described by \cite{GalBachelis1991} and uses accurate tables produced by the method presented in \cite{StehléZimmermann2005}.  It guarantees correct rounding with a high probability.  In circumstances where it cannot guarantee correct rounding, it falls back to the (slower but correct) implementation provided by the CORE-MATH project \cite{SibidanovZimmermannGlondu2022} \cite{ZimmermannSibidanovGlondu2024}.  More precisely, the algorithm proceeds through the following steps:
\begin{itemize}[nosep]
\item perform argument reduction using Cody and Waite's algorithm in double precision (see \cite[379]{MullerBrisebarreDeDinechinJeannerodLefevreMelquiondRevolStehleTorres2010});
\item if argument reduction loses too many bits (i.e., the argument is close to a multiple of $\frac{\Pi}{2}$), fall back to \texttt{cr\_sin} or \texttt{cr\_cos};
\item otherwise, uses accurate tables and a polynomial approximation to compute \texttt{Sin} or \texttt{Cos} with extra accuracy;
\item if the result has a ``dangerous rounding configuration'' (as defined by \cite{GalBachelis1991}), fall back to \texttt{cr\_sin} or \texttt{cr\_cos};
\item otherwise return the rounded result of the preceding computation.
\end{itemize}
In this document we assume a base-2 floating-point number system with $M$ significand bits\footnote{In \texttt{binary64}, $M = 53$.} similar to the IEEE formats.  We define a real  function $\mant$ and an integer function $\expn$ denoting the \emph{significand} and \emph{exponent} of a real number, respectively:
\[
x = ±\mant\of{x} \times 2^{\expn\of{x}} \qquad\text{with}\qquad 2^{M-1} \leq \mant\of{x} \leq 2^M - 1
\]
Note that this representation is unique.  Furthermore, if $x$ is a floating-point number, $\mant\of{x}$ is an integer.

The distance between $1$ and the next larger floating-point number is:
\[
\ge_M \DefineAs 2^{1-M}
\]
and the distance between $1$ and the next smaller floating-point number is $\frac{\ge_M}{2}$.
The \emph{unit of the last place} of $x$ is defined as:
\[
\ULP\of{x} \DefineAs 2^{\expn\of{x}}
\]
In particular, $\ULP\of{1} = \ge_M$.

We ignore the exponent bias, overflow and underflow as they play no role in this discussion.
\section*{Argument Reduction}
Given an argument $x$, the purpose of argument reduction is to compute a pair of floating-point numbers $\pa{\hat x, \gd \hat x}$ such that:
\[
\begin{dcases}
\hat x + \gd \hat x ≅ x \pmod{\frac{\Pi}{2}} \\
\hat x \;\text{is approximately in}\; \intclos{-\frac{\Pi}{4}}{\frac{\Pi}{4}} \\
\abs{\gd \hat x} < \ULP\of{\hat x} 
\end{dcases}
\]
\subsection*{Approximation of $\Pi$}
We approximate $\frac{\Pi}{2}$ as the sum of two floating-point numbers:
\[
\frac{\Pi}{2} ≅ C + \gd C
\]
where $C$ is obtained by truncating $\frac{\Pi}{2}$ to $\gk_1$ significand bits:
\[
C \DefineAs \floor{2^{-\gk_1} \mant \of{\frac{\Pi}{2}}} 2^{\gk_1} \ULP\of{\frac{\Pi}{2}}
\]
and $\gd C$ is defined as $\round{\frac{\Pi}{2} - C}$.  Obviously we have:
\[
0 < \frac{\Pi}{2} - C < 2^{\gk_1} \ULP\of{\frac{\Pi}{2}}
\]
but if $\gk_1$ is chosen to cut the significand of $\frac{\Pi}{2}$ at a place where it has zeroes, we can actually have a stricter bound:
\begin{align}
\frac{\Pi}{2} - C < 2^{\gk_2} \ULP\of{\frac{\Pi}{2}} \qquad\text{with}\qquad \gk_2 \leq \gk_1
\label{eqndC}
\end{align}
and therefore:
\[
\ULP\of{\frac{\Pi}{2} - C} < \frac{2^{\gk_2} \ULP\of{\frac{\Pi}{2}}}{\mant\of{\frac{\Pi}{2} - C}} \leq 2^{\gk_2 - M + 1} \ULP\of{\frac{\Pi}{2}} 
\]
Since the function $\ULP$ is always a power of $2$ this implies:
\[
\ULP\of{\frac{\Pi}{2} - C} = 2^{\gk_2 -M} \ULP\of{\frac{\Pi}{2}}
\]
and:
\begin{align}
\abs{\frac{\Pi}{2} - C - \gd C} \leq \frac{1}{2} \ULP\of{\frac{\Pi}{2} - C} = 2^{\gk_2 - M - 1} \ULP\of{\frac{\Pi}{2}}
\label{eqnpi}
\end{align}
In other words, we have a representation with a significand that has effectively $2 M - \gk_2$ bits and is such that multiplying $C$ by an integer less than or equal to $2^{\gk_1}$ is exact.  The representation of $\frac{\Pi}{2}$ has three zeroes after the 18th bit of its significand, so by taking $\gk_1 = 18$ we have $\gk_2 = 14$.

\subsection*{Argument Reduction for Small Angles}
If $\abs x < \round{\frac{\Pi}{4}}$ then $\hat x = x$ and $\gd \hat x = 0$.
\subsection*{Argument Reduction for Medium Angles}
If $\abs x \leq 2^{\gk_1} \round{\frac{\Pi}{2}}$ then we compute:
\[
\begin{dcases}
n &= \iround{\round{x \round{\frac{2}{\Pi}}}} \\
y &= x - n \; C \\
\gd y &= \round{n \; \gd C} \\
\hat x &= \round{y - \gd y} \\
\gd \hat x &= \pa{y - \hat x} - \gd y
\end{dcases}
\]
First, note that $\abs n \leq 2^{\gk_1}$.  Using the accuracy model of \cite{Higham2002}, equation (2.4), we have\footnote{We note that in Higham's notation $u = \ge_M / 2$, see pages 37 and 38.}:
\begin{align*}
\abs x &\leq 2^{\gk_1} \frac{\Pi}{2} \pa{1 + \gd_1} \\
\abs n &\leq \iround{2^{\gk_1} \frac{\Pi}{2} \pa{1 + \gd_1} \frac{2}{\Pi} \pa{1 + \gd_2} \pa{1 + \gd_3}} \\
&\leq \iround{2^{\gk_1} \pa{1 + \gg_3}}
\end{align*}
where the notation follows \cite{Higham2002}, lemma 3.1.  Because $2^{\gk_1} \gg_3$ is very small (less that $2^{-33}$), the rounding cannot cause $n$ to exceed $2^{\gk_1}$.

The product $n \; C$ is exact thanks to the $\gk_1$ trailing zeroes of $C$.  The subtraction $x - n \; C$ is exact by Sterbenz's Lemma.  Finally, the last two steps form a compensated summation so that $\hat x + \gd \hat x = y + \gd y$.

To compute the overall error on argument reduction, first remember that, from equation \ref{eqnpi} we have:
\[
C + \gd C = \frac{\Pi}{2} + \gd_5 \quad \text{with} \quad \abs{\gd_5} \leq 2^{\gk_2 - M - 1} \ULP\of{\frac{\Pi}{2}}
\]
The error computation proceeds as follows:
\begin{align*}
y + \gd y &= x - n \; C - n \; \gd C \pa{1 + \gd_4}  \quad \text{with} \quad \abs{\gd_4} < 2^{-M}\\
&= x - n \pa{C + \gd C} - n \; \gd C \; \gd_4 \\
&= x - n \frac{\Pi}{2} -n \pa{\gd_5 + \gd C \; \gd_4}
\end{align*}
from which we can deduce an upper bound of the error:
\[
\abs{y + \gd y - \pa{x - n \frac{\Pi}{2}}} < 2^{\gk_1} 2^{\gk_2} \ULP\of{\frac{\Pi}{2}} \pa{2^{- M - 1} + 2^{-M}} = \frac{3}{2} 2^{\gk_1 + \gk_2 - M} \ULP\of{\frac{\Pi}{2}}
\]
where we have used the upper bound on $\gd C$ given by equation \ref{eqndC}.

This means that the reduction is only correctly rounded to $M$ bits if $\abs{\hat x} \geq \frac{3}{2} 2^{\gk_1 + \gk_2} \ULP\of{\frac{\Pi}{2}}$.  If more bits are cancelled, a more advanced technique is needed.

\emph{TODO(phl):} It is not clear how many bits can actually be cancelled in this range.  Use Muller's program to figure out.  If the number is small enough, we may not even need to fall back to CORE-MATH in that case.

\subsection*{Argument Reduction for Large Angles}
\section*{Accurate Tables and Their Generation}
\section*{Computation of the Functions}

\end{document}
